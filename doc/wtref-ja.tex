\documentclass[a4paper,uplatex]{jsarticle}

\usepackage{otf}
\usepackage{okumacro}
\usepackage{shortvrb}
\MakeShortVerb{\|}
\newcommand{\Meta}[1]{$\langle$\mbox{}#1\mbox{}$\rangle$}
\newcommand{\Note}{\par\noindent ※}
\newenvironment{syntax}{\begin{quote}\small}{\end{quote}}

\title{\texttt{wtref.sty} (v0.0)}
\author{ワトソン}

\begin{document}

\maketitle

\begin{abstract}
\textsf{WTStyles}パッケージは著者が\LaTeX 文書作成にあたってよく利用するマクロを集めた
ものである.\texttt{wtref.sty}はこの\textsf{WTStyles}パッケージを構成する要素の1つであり,
\LaTeX の相互参照機能を拡張してスコープの指定と名前空間の分離,参照形式の柔軟な指定を可能
にする.任意の\TeX エンジンと\LaTeXe の組み合わせで動作する.
\end{abstract}

\section{パッケージ読み込み}

|\usepackage| 命令を用いて読み込む.この際,\texttt{wtref.sty}が管理する相互参照命令(後述)に
よって作成されるラベルの有効範囲(スコープ)をオプションとして指定することができる.
%
\begin{syntax}
|\usepackage[|\Meta{スコープ}|]{wtref}|
\end{syntax}

オプションの\Meta{スコープ}に指定できる値は以下の4つである.ただし,パッケージオプションに
何も指定しない場合は\texttt{global}が設定される.
%
\begin{itemize}
\item \texttt{chapter}:スコープを章単位に設定する.|\chapter| 命令の存在しない環境で
指定するとエラーになる.
\item \texttt{section}:スコープを節単位に設定する.
\item \texttt{subsection}:スコープを小節単位に設定する.
\item \texttt{global}:スコープを設定しない(デフォルト).
\end{itemize}

\section{相互参照命令}

\subsection{相互参照命令の新設}

|\newref| 命令を用いて相互参照に用いる2つの命令の組を新設することができる.
%
\begin{syntax}
|\newref{|\Meta{名前}|}|
\end{syntax}
%
ただし,この |\newref| 命令はプリアンブルでしか使用できない.また\Meta{名前}に使えるのは
原則として制御綴に使用できる文字のみであり(半角英字のみにしておくことを推奨),空であっては
ならない.

|\newref| 命令は指定した\Meta{名前}に応じて |\|\Meta{名前}|label|,|\|\Meta{名前}|ref| という
形をもつ新しい2つの命令を定義する.すなわち,プリアンブル中で
%
\begin{syntax}
|\newref{ex}|
\end{syntax}
%
のように宣言を行うと,|\exlabel| と |\exref| という命令が新たに定義される.ここで前者の命令を
\textbf{ラベル命令},後者の命令を\textbf{参照命令}と呼ぶことにする.

\subsection{ラベル命令}

\subsubsection{機能と使い方}

ラベル命令はラベルを付けるときに用いる.使い方は\LaTeX の |\label| 命令とまったく同様である.
以下にラベル命令の例 |\exlabel| を用いるときの書式を示しておく.
%
\begin{syntax}
|\exlabel{|\Meta{ラベル}|}|
\end{syntax}

\subsubsection{内部処理}

ラベル命令は最終的に次のような形に展開される.
%
\begin{syntax}
|\label{|\Meta{名前}|:|\Meta{スコープ番号}|:|\Meta{ラベル}|}|
\end{syntax}

ここで\Meta{スコープ番号}はパッケージオプションで指定した\Meta{スコープ}に依存して決定される
アラビア数字とピリオドの組み合わせである.各指定\Meta{スコープ}ごとの具体的な形を以下に
列挙する.
%
\begin{itemize}
\item \texttt{chapter}:``\Meta{章番号}''
\item \texttt{section}:``\Meta{章番号}.\Meta{節番号}''または``\Meta{節番号}''
\item \texttt{subsection}:``\Meta{章番号}.\Meta{節番号}.\Meta{小節番号}''または
``\Meta{節番号}.\Meta{小節番号}''
\end{itemize}

ただし\Meta{スコープ}が\texttt{global}に設定されている場合は次の形に展開される.
%
\begin{syntax}
|\label{|\Meta{名前}|:|\Meta{ラベル}|}|
\end{syntax}

\subsection{参照命令}

参照命令は対応するラベル命令によってラベル付けされた番号を,指定した書式で印字するための
命令である.例として |\exref| の使い方を以下に示す.
%
\begin{syntax}
|\exref[|\Meta{スコープ番号}|]{|\Meta{ラベル1}|,|\Meta{ラベル2}|,...}|
\end{syntax}

参照するラベルが同じスコープ内に存在する場合は\Meta{スコープ番号}は省略可能である.特に
\Meta{スコープ}が\texttt{global}に設定されている場合は常に省略可能であり,逆にオプション
引数に値を指定しても無視されるだけである.

また,引数にはカンマ区切りで複数の参照先ラベルを指定することが可能である.複数のラベルを指定
した場合,デフォルトでは該当する番号がカンマ区切りで出力される.この出力書式は後述する
参照書式変更命令 |\setrefstyle| を用いて柔軟にカスタマイズすることができる.

\subsection{式および図表のための相互参照命令}

\texttt{wtref.sty}を読み込むと,次の命令が自動的に用意される.
%
\begin{itemize}
\item |\eqlabel| と |\eqref|:式のための相互参照命令
\item |\figlabel| と |\figref|:図のための相互参照命令
\item |\tablabel| と |\tabref|:表のための相互参照命令
\end{itemize}

ここで |\eqref| は最初から式番号を半角の括弧で囲うように設定されている.それ以外は
前述した参照命令のデフォルト設定にしたがう.

\section{参照書式変更命令}

|\setrefstyle| 命令を用いると参照命令の出力を柔軟にカスタマイズすることができる.
最初に |\setrefstyle| 命令の書式を示しておく.
%
\begin{syntax}
|\setrefstyle{|\Meta{名前}|}{|\Meta{区切り}|}[|\Meta{最後の区切り}|]{|\Meta{接頭辞1}|}{|%
\Meta{接頭辞2}|}{|\Meta{接尾辞2}|}{|\Meta{接尾辞1}|}|
\end{syntax}

これにより\Meta{名前}に対応する参照命令の書式を変更できる.このとき既に設定されている書式は
すべて上書きされるので注意されたい.この |\setrefstyle| 命令はプリアンブルに限らず\LaTeX 文書
ソース中全域で使用可能であり,|{| や |}| によるブロックの制御を受ける.

\Meta{区切り}は各参照番号の区切りに用いられる文字列を指定する.ここで\Meta{最後の区切り}を
指定した場合には,最後の区切りにはその文字列が用いられる.\Meta{接頭辞1}は該当する参照命令を
用いると必ず先頭にただ1度だけ出力される.\Meta{接頭辞2}は各参照番号の直前に置かれる文字列
である.逆に,\Meta{接尾辞2}は各参照番号の直後に置かれる文字列である.\Meta{接尾辞1}は
参照命令を用いると,当該命令が出力する内容の末尾にただ1度だけ出力される.







\end{document}
